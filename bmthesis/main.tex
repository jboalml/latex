% ---------------------------------------------------------------------
% ---------------------- PROYECTO FIN DE CARRERA ----------------------
% ---------------------------------------------------------------------
%
% Archivos: ProyectoFinCarrera.tex
%
% Descripción: Plantilla para escribir el Proyecto Fin de Carrera.
%
% Versión: 1.0.0
%
% Fecha: 30/04/2010
%
% Autor: Jaime Boal Martín-Larrauri
%
% ---------------------------------------------------------------------
% ----------------------------- PARÁMETROS ----------------------------
% ---------------------------------------------------------------------

	\newcommand{\Asunto}{Proyecto Fin de Carrera}
	\newcommand{\Autor}{Nombre del Autor}
	\newcommand{\Centro}{Escuela Técnica Superior de Ingeniería (ICAI)}
	\newcommand{\Coordinador}{Nombre del Coordinador}
	\newcommand{\Director}{Nombre del Director}
	\newcommand{\Fecha}{Junio de 2010}
	\newcommand{\Keywords}{}
	\newcommand{\PalabrasClave}{}
	\newcommand{\Universidad}{Universidad Pontificia Comillas}
	\newcommand{\Title}{Final Project Title}
	\newcommand{\Titulacion}{Ingeniero Industrial}
	\newcommand{\Titulo}{Título oficial del Proyecto Fin de Carrera}

% ---------------------------------------------------------------------
% ----------------------------- PREÁMBULO -----------------------------
% ---------------------------------------------------------------------

% 1. PROPIEDADES DEL DOCUMENTO

% Clase de documento
\documentclass[12pt,english,spanish]{bmthesis}
		
\Author{Author}
\Degree{Degree}
		
\begin{comment}
	% Determina si el documento se compila en .pdf	
		\usepackage{ifpdf}
		
% ---------------------------------------------------------------------
	
% 2. IDIOMAS

		\usepackage[english, spanish]{babel}	% Idiomas cargados: inglés y español (por defecto)
		
	% Encabezados en castellano
		\addto\captionsspanish{
			\def\contentsname{Índice general}
			\def\listtablename{Índice de tablas}
			\def\tablename{Tabla}
		}

% ---------------------------------------------------------------------

% 3. FUENTES

	% Codificación
		\usepackage[latin1]{inputenc} % Poner acentos directamente
		\usepackage[T1]{fontenc}      % Separación por guiones correcta
																	
	% Separación correcta de palabras (sólo si es necesario) 	
		%\hyphenation{hy-phen-a-tion}
	
	% Fuentes seleccionadas
		\usepackage{times}	  % Fuente predeterminada
		\usepackage{amsfonts} % Fuente de la American Mathematical Society
		\usepackage{courier}  % Fuente monoespaciada para código
		\usepackage{alltt}    % Entorno de código sencillo que permite comandos de LaTeX
		
	% Otras fuentes
		%\usepackage{lmodern}  % Fuente por defecto
		%\usepackage{pxfonts}  % Similar a la times, pero más antigua
		%\usepackage{chancery} % Fuente cursiva de estilo antiguo
		%\usepackage{mathptmx} % Fuente del entorno matemático
		
	% Títulos
		\usepackage{sectsty}
			\allsectionsfont{\fontfamily{lmss}\selectfont}
	
	% Letras capitales
		\usepackage{lettrine}
			\setlength{\DefaultFindent}{0.2em}	% Separación letra - 1ª línea del bloque de texto
			\setlength{\DefaultNindent}{0em}		% Separación letra - bloque de texto
			\setcounter{DefaultLines}{2}				% Número de líneas que ocupa la letra capital 

% ---------------------------------------------------------------------

% 4. FORMATO

	% Márgenes
		\usepackage[left=3cm, right=2cm, top=2.5cm, bottom=2cm]{geometry}
		%\usepackage {anysize} 								% Paquete alternativo
			%\marginsize {2cm}{2cm}{1cm}{1.5cm}	% Izquierdo, derecho, superior, inferior
		\usepackage{changepage}								% Permite cambiar de forma relativa los márgenes de páginas concretas
		
	% Marcas de agua	
		\usepackage{watermark}			
		
	% Formato de las páginas de Parte
		\usepackage[newparttoc]{titlesec}
		\usepackage{titletoc}	
			\titleformat{\part}[display]
									{\vspace*{\fill}\fontsize{36}{40}\fontfamily{ppl}\bfseries\selectfont}
									{\filcenter\fontsize{30}{32}\selectfont\MakeUppercase{\partname}\hspace{8pt}\thepart\\
									\hspace{7pt}\rule{75pt}{2.5pt}\vspace{15pt}						% Reemplazar esta línea por la siguiente para sustituir
									%\hspace{7pt}\fontsize{36}{40}\selectfont\ding{167}		% el icono del paquete pifonts por una línea y viceversa
									\thiswatermark{\centering\put(268,-802){\includegraphics[scale=1.5]{ComillasWatermark.pdf}}}}
									{0mm}
									{\filcenter\MakeUppercase}
									[\vspace*{\fill}\thispagestyle{empty}\setcounter{chapter}{0}]
									
	% Formato de otras divisiones: Párrafo y Subpárrafo								
			\titleformat{\paragraph}
									{\fontfamily{lmss}\selectfont\normalsize\bfseries}{\theparagraph .}{1em}{}
			\titlespacing*{\paragraph}{0pt}{3.25ex plus 1ex minus .2ex}{1.5ex plus .2ex}
															
			\titleformat{\subparagraph}
									{\fontfamily{lmss}\selectfont\normalsize\bfseries}{\thesubparagraph .}{1em}{}
			\titlespacing*{\subparagraph}{0pt}{3.25ex plus 1ex minus .2ex}{1.5ex plus .2ex}
			
	% Formato de capítulo: Extendido y Compacto	
			\makeatletter
				\newcommand{\CapituloExtendido}
									 {\renewcommand\chapter{\if@openright\cleardoublepage\else\clearpage\fi
									 \thispagestyle{plain}
									 \rightside{\thechapter . \leftmark}
									 \global\@topnum\z@
									 \@afterindentfalse
									 \secdef\@chapter\@schapter}
									 \titleformat{\chapter}[display]{\fontfamily{lmss}\huge\bfseries\selectfont}
		 													 {\chaptertitlename\ \thechapter}{20pt}{\Huge} 
		 							 \titlespacing*{\chapter}{0pt}{50pt}{40pt}}
			\makeatother
						
			\makeatletter
				\newcommand{\CapituloCompacto}	% NOTA: Hay que controlar los saltos de página de este tipo de capítulos manualmente
									 {\renewcommand\chapter{\par
									 \thispagestyle{fancy}
									 \rightside{\thechapter . \leftmark}
									 \global\@topnum\z@
									 \@afterindentfalse
									 \secdef\@chapter\@schapter}
									 \titleformat{\chapter}{\fontfamily{lmss}\LARGE\bfseries\selectfont}{\thechapter .}{1em}{} 
									 \titlespacing*{\chapter}{0pt}{3.5ex plus 1ex minus .2ex}{2.3ex plus .2ex}}
			\makeatother
																		 						
	% Crear el tipo de división doc (para Documentos)					
		\makeatletter														
			\newcounter{documento}
			\newcommand{\thedocument}{\Roman{documento}}
		\makeatother					 
		
		\newcommand{\documentname}{Documento}														
		\newcommand{\doc}[1]
							 {\addtocounter{documento}{1}\clearpage\phantomsection\addcontentsline{toc}{part}
							 {\texorpdfstring{\uppercase{\documentname \ \thedocument.\hspace{8pt}#1}}{\documentname \ \thedocument.\hspace{8pt}#1}}
							 \thispagestyle{empty}\begin{center}
							 {\vspace*{\fill}\fontsize{36}{32}\fontfamily{ppl}\bfseries\selectfont
							 \MakeUppercase{\documentname}\hspace{10pt}\thedocument \\ 
							 \rule{75pt}{2.5pt} \\											% Reemplazar esta línea por la siguiente para sustituir 
							 %\fontsize{50}{54}\selectfont\ding{166} \\	% el icono del paquete pifonts por una línea y viceversa
							 \fontsize{40}{44}\selectfont \linespread{1.4}\vspace{0.4cm} \MakeUppercase{#1}
							 \thiswatermark{\centering\put(268,-802){\includegraphics[scale=1.5]{ComillasWatermark.pdf}}}\\
							 \vspace*{\fill}}\end{center}\setcounter{part}{0}\setcounter{chapter}{0}\setcounter{page}{1}
							 \setcounter{figure}{0}\setcounter{table}{0}\setcounter{code}{0}\setcounter{footnote}{0}\setcounter{equation}{0}}

	% Crear el tipo de división hc (para insertar una página de encabezado en las Hojas de características)								 
		\newcommand{\hc}[1]
							 {\clearpage\phantomsection\addcontentsline{toc}{chapter}{#1}
							 \thispagestyle{empty}\begin{center}						 
							 {\vspace*{\fill}\begin{onehalfspace}\fontsize{26}{34}\fontfamily{ppl}\bfseries\selectfont 
							 #1 \end{onehalfspace}
							 \thiswatermark{\centering\put(195,-945){\includegraphics[scale=.25, angle = 25]{ICAIWatermark.pdf}}}
							 \vspace*{\fill}}\end{center}}
							 
	% Sobrecarga de divisiones
		\let\Documento\doc
		\renewcommand{\doc}[1]{\clearpage\Documento{#1}\leftside{\documentname \ \thedocument. #1}}
	
		\let\Parte\part
		\renewcommand{\part}[3]{\label{#2}\clearpage\label{#3}\Parte{#1}\leftside{\thepart. #1}}
				
		\let\Capitulo\chapter	
		\renewcommand*{\chapter}{\secdef{\CapituloNumerado}{\CapituloNoNumerado}}
			\newcommand\CapituloNumerado[2][]{\Capitulo[#1]{#2}\rightside{\thechapter. #2}} % Alternativa: \rightside{\thechapter . \leftmark}
			\newcommand\CapituloNoNumerado[1]{\Capitulo*{#1}\rightside{#1}}
								 
	% Formato de los índices general, de figuras y de tablas
		\usepackage{tocloft}
		
			% Numeración de secciones
				\setcounter{secnumdepth}{5} % Último nivel numerado: subparagraph (máx. 5)
				\setcounter{tocdepth}{5}		% Último nivel mostrado en los índices: subparagraph (máx. 5)
	
			% Márgenes
				\setlength{\cftbeforetoctitleskip}{63.5pt}	% Espacio que precede al título del índice general
				\setlength{\cftbeforeloftitleskip}{63.5pt}	% Espacio que precede al título del índice de figuras
				\setlength{\cftbeforelottitleskip}{63.5pt}	% Espacio que precede al título del índice de tablas
				\setlength{\cftbeforepartskip}{12pt}				% Espacio que precede a las partes
				\setlength{\cftbeforechapskip}{6pt}					% Espacio que precede a los capítulos
				%\cftsetpnumwidth{1.25em} 									% Espacio para el número de página
				%\cftsetrmarg{1.4em}      									% Distancia del texto al margen derecho (mayor que la anterior)
				\renewcommand{\cftpartnumwidth}{2.2em} 			% Espacio para el número de parte
				\renewcommand{\cftchapnumwidth}{1.6em} 			% Espacio para el número de parte
				\renewcommand{\cftfignumwidth}{1.6em} 			% Espacio para el número de figura
				\renewcommand{\cfttabnumwidth}{1.6em} 			% Espacio para el número de tabla
				%\renewcommand{\cftpartnumwidth}{2.3em}			% Espacio para el número de parte
				%\setlength{\cftsecindent}{0pt}	      			% Sangría del número de sección
				\setlength{\cftsubsecindent}{26.65pt}	      % Sangría del número de subsección
				\setlength{\cftsubsubsecindent}{35.70pt}	  % Sangría del número de subsubsección
				\setlength{\cftparaindent}{44.75pt}	      	% Sangría del número de párrafo
				\setlength{\cftsubparaindent}{53.80pt}	    % Sangría del número de subpárrafo
				\setlength{\cftfigindent}{0pt}        			% Sangría del número de figura
				\setlength{\cfttabindent}{0pt}	      			% Sangría del número de tabla
				
			% Texto que precede al número de cada sección
				%\renewcommand{\cftpartpresnum}{Parte }			% Interacciona con algo y sale duplicado
				%\renewcommand{\cftfigfont}{Figura }
				%\renewcommand{\cfttabfont}{Tabla }
			
			% Texto posterior al número de cada sección	
				\renewcommand{\cftpartaftersnum}{.}
				\renewcommand{\cftchapaftersnum}{.}
				\renewcommand{\cftsecaftersnum}{.}
				\renewcommand{\cftsubsecaftersnum}{.}
				\renewcommand{\cftsubsubsecaftersnum}{.}
				\renewcommand{\cftparaaftersnum}{.}
				\renewcommand{\cftsubparaaftersnum}{.}
				\renewcommand{\cftfigaftersnum}{.}
				\renewcommand{\cfttabaftersnum}{.}
								
			% Separación de los puntos suspensivos
				%\renewcommand{\cftpartdotsep}{3.5}
				%\renewcommand{\cftchapdotsep}{3.5}
				%\renewcommand{\cftsecdotsep}{0.5}
				%\renewcommand{\cftfigdotsep}{0.5}
				%\renewcommand{\cfttabdotsep}{0.5}
				
			% Fuente de los títulos, las secciones y los números de página
				\renewcommand{\cfttoctitlefont}{\fontfamily{lmss}\huge\bfseries\selectfont}
				\renewcommand{\cftloftitlefont}{\fontfamily{lmss}\huge\bfseries\selectfont}
				\renewcommand{\cftlottitlefont}{\fontfamily{lmss}\huge\bfseries\selectfont}
				\renewcommand{\cftpartfont}{\fontfamily{ppl}\large\bfseries\selectfont}
				\renewcommand{\cftpartpagefont}{\fontfamily{ppl}\large\bfseries\selectfont}
				%\renewcommand{\cftchapfont}{\fontfamily{ppl}\bfseries\selectfont}
				%\renewcommand{\cftchappagefont}{\fontfamily{ppl}\bfseries\selectfont}
								
% ---------------------------------------------------------------------

% 5. ENCABEZADO Y PIE DE PÁGINA

	% Estilo
		\usepackage{fancyhdr}
			\pagestyle{fancy}

	% Centrar en la zona del margen
		\setlength{\voffset}{-13pt}
		\setlength{\headheight}{13pt}
		\footskip = 36pt

	% Formato de las líneas separadoras
		\renewcommand{\headrulewidth}{0pt} 
		\renewcommand{\footrulewidth}{0.4pt}
		%\renewcommand{\footrule}{{\color{blue}
		%							 \vskip-\footruleskip\vskip-\footrulewidth\hrule width\headwidth height\footrulewidth\vskip\footruleskip}}
	
	% Formato del texto
		\fancyhf{} 
		\renewcommand{\chaptermark}[1]{\markboth{#1}{}}
		\renewcommand{\sectionmark}[1]{\markright{#1}{}}
		\newcommand{\theleftside}{}
		\newcommand{\leftside}[1]{\renewcommand{\theleftside}{#1}}
		\newcommand{\therightside}{}
		\newcommand{\rightside}[1]{\renewcommand{\therightside}{#1}}
		\newcommand{\EncabezadoCorto}{\fancyhead[C]{\footnotesize{\textsc{\theleftside}}}}
		\newcommand{\EncabezadoLargo}{\fancyhead[C]{\footnotesize{\textsc{\theleftside \ \ \ding{167} \ \therightside}}}}
		\fancyfoot[R]{\thepage}
		\fancyfoot[L]{\footnotesize{\textit{\Titulo}\\ \textbf{\Autor}}}
		
	% Personalizar las páginas de capítulo, índice...	
		\fancypagestyle{plain}{
			\fancyhf{}
			\fancyfoot[R]{\thepage}
			\fancyfoot[L]{\footnotesize{\textit{\Titulo}\\ \textbf{\Autor}}}
			%\renewcommand{\footrule}{{\color{blue}
			%							 \vskip-\footruleskip\vskip-\footrulewidth\hrule width\headwidth height\footrulewidth\vskip\footruleskip}}
			\renewcommand{\headrulewidth}{0pt}
			\renewcommand{\footrulewidth}{0.4pt}
		}

% ---------------------------------------------------------------------

% 6. NOTAS AL PIE
		
	% Numerar correlativamente las notas al pie a lo largo de todo el documento  	
		\usepackage{remreset} 
			\makeatletter
				\@removefromreset{footnote}{chapter}
			\makeatother

% ---------------------------------------------------------------------

% 7. FIGURAS, TABLAS, ECUACIONES, LISTAS Y CUADROS DE TEXTO 
	
		\usepackage{sidecap}		% Colocar pies de foto al lado del texto

  % Numerar correlativamente las figuras, tablas y ecuaciones a lo largo de todo el documento  
	  \usepackage{remreset}
	 		\makeatletter
				\@removefromreset{figure}{chapter}
				\renewcommand{\thefigure}{\arabic{figure}}
				\@removefromreset{table}{chapter}
				\renewcommand{\thetable}{\arabic{table}}
				\@removefromreset{equation}{chapter}
				\renewcommand{\theequation}{\arabic{equation}}
			\makeatother
			
	% Formato de la numeración de las ecuaciones
		%\numberwithin{equation}{chapter}	% Numerar por capítulos
		
	% Tablas	
		\usepackage{colortbl}		% Para crear tablas con filas y columnas coloreadas
		
	% Cuadro de texto
		\newlength\Linewidth
			\def\findlength{
				\setlength\Linewidth\linewidth
				\addtolength\Linewidth{-4\fboxrule}
				\addtolength\Linewidth{-3\fboxsep}
			}
		\newenvironment{cuadrotexto}{
				\par\begingroup
				\setlength{\fboxsep}{5pt}\findlength
				\setbox0=\vbox\bgroup\noindent
				\hsize=\Linewidth
				\begin{minipage}{\Linewidth}
			}{
				\end{minipage}\egroup
				\vspace{6pt}
				\noindent\textcolor{gris20}{\fboxsep2.5pt\fbox
				{\fboxsep5pt\colorbox{gris05}{\normalcolor\box0}}}
				\endgroup\par\addvspace{6pt minus 3pt}\noindent
				\normalcolor\ignorespacesafterend
			}

% ---------------------------------------------------------------------

% 10. OTROS

		%\usepackage{siunitx}									% Para introducir las unidades con la separación adecuada
		%\usepackage[cdot, squaren]{SIunits}	% Otro paquete para escribir unidades (mejor pero más complejo de usar)
		\usepackage{lipsum}										% Texto lorem ipsum en latín para comprobar la tipografía y la apariencia.
																					% NOTA: Para que funcione hay que sustituir en lipsum.sty \roman{lips@count} 				
																					% por \romannumeral\value{lips@count}

% ---------------------------------------------------------------------

% 11. ENLACES, REFERENCIAS CRUZADAS Y NUEVOS ÍNDICES DE CONTENIDO

		\usepackage[pdftex]{hyperref} % Siempre es el último paquete que se carga
			\hypersetup{
				pdffitwindow = true,
		    pdftitle = \Titulo,
		    pdfauthor = \Autor,
		    pdfsubject = \Asunto,
		    pdfkeywords = \PalabrasClave,
		    pdfnewwindow = true,
		    colorlinks = true,
		    citecolor = black,
		    filecolor = black,
		    linkcolor = black,
		    urlcolor = blue,
		    %plainpages = false,					% Impedir referencias de página con el mismo nombre
		    hypertexnames = false,				% Impedir referencias con el mismo nombre
		    bookmarksnumbered = true,
		    bookmarksopen = true,
		    bookmarksopenlevel = -1
			}
			
			\addto\captionsspanish{
				\renewcommand{\partautorefname}{Parte}
				\renewcommand{\chapterautorefname}{Capítulo}
				\renewcommand{\sectionautorefname}{Apartado}
				\renewcommand{\figureautorefname}{Figura}
				\renewcommand{\tableautorefname}{Tabla}
				\renewcommand{\equationautorefname}{Ecuación}
			}
				
	% Redefinir los niveles de las divisiones del documento para poder incorporar doc en el superior.
	% Para ello hay que añadir al final de Proyecto Fin de Carrera.out la linea \let\WriteBookmarks\relax
	% después de compilar la versión definitiva (Debe ir detrás de hyperref)
		\makeatletter
			\renewcommand*{\toclevel@part}{0}
			\renewcommand*{\toclevel@chapter}{1}
			\renewcommand*{\toclevel@section}{2}
			\renewcommand*{\toclevel@subsection}{3}
			\renewcommand*{\toclevel@subsubsection}{4}
			\renewcommand*{\toclevel@paragraph}{5}
			\renewcommand*{\toclevel@subparagraph}{6}
		\makeatother
	
	% Índice de extractos de código si se usa el paquete caption (detrás de hyperref)
	 	\usepackage{float}
 		\newlistof{codigo}{loc}{Índice de extractos de código}
			\newlistentry{code}{loc}{0}
			\newfloat{code}{htbp}{loc}
			\floatname{code}{Código}
		
			%\renewcommand{\cftcodefont}{Código }
		 	\renewcommand{\cftcodeaftersnum}{.}
		 	\setlength{\cftcodeindent}{0pt}        % Sangría del número de código
		 	\renewcommand{\cftcodenumwidth}{1.6em} % Espacio para el número de código
		 	\renewcommand{\cftafterloctitle}{\thispagestyle{plain}}
		 	\setlength{\cftbeforeloctitleskip}{63.5pt}
		 	\renewcommand{\cftloctitlefont}{\fontfamily{lmss}\huge\bfseries\selectfont}
		 	
\end{comment}
			 			 	
% ---------------------------------------------------------------------

% 12. MACROS (PARA FACILITAR EL USO DE LA PLANTILLA)
			
		% Insertar una línea de separación entre párrafos		
				\newcommand{\IndiceGeneral}{
										\clearpage
										\cftpagenumberson{chapter}
										\pdfbookmark[0]{Índice general}{General}
										\tableofcontents
										\leftside{Índice general}}
										
				\newcommand{\IndiceDocumento}[2]{
										\clearpage\rightside{#1}\startcontents[toc]
										\cftpagenumberson{chapter}
										\renewcommand{\cftchappresnum}{}
										\renewcommand{\cftchapnumwidth}{1.6em}
										\renewcommand{\cftchapfont}{\normalfont\bfseries}
										\setcounter{tocdepth}{5}
										\pdfbookmark[0]{#1}{#2}		
										\printcontents[toc]{}{1}{\chapter*{#1}}
										\label{#2}
										\clearpage}
										
				\newcommand{\ListaPlanos}[2]{
										\clearpage\rightside{#1}\startcontents[toc]
										\cftpagenumbersoff{chapter}
										\renewcommand{\cftchappresnum}{Plano }
										\renewcommand{\cftchapnumwidth}{4.2em}
										\renewcommand{\cftchapfont}{\normalfont}
										\setcounter{tocdepth}{0}
										\pdfbookmark[0]{#1}{#2}		
										\printcontents[toc]{}{1}{\chapter*{#1}}
										\label{#2}
										\clearpage}
										
				\newcommand{\FinIndiceDocumento}{
										\stopcontents[toc]}
				
				\newcommand{\IndiceFiguras}{	
										\clearpage
										\pdfbookmark[0]{Índice de figuras}{Figuras}
										\listoffigures
										\rightside{Índice de figuras}}
										
				\newcommand{\IndiceTablas}{
										\clearpage
										\pdfbookmark[0]{Índice de tablas}{Tablas}
										\listoftables
										\rightside{Índice de tablas}}
										
				\newcommand{\IndiceCodigo}{
										\clearpage
										\pdfbookmark[0]{Índice de extractos de código}{Codigo}
										\listofcodigo
										\rightside{Índice de extractos de código}}
										
				\newcommand{\ActualizarIndices}{
										\captionsetup[figure]{list=yes}
										\captionsetup[table]{list=yes}
										\captionsetup[code]{list=yes}}						
										
				\newcommand{\SuprimirIndices}{
										\captionsetup[figure]{list=no}
										\captionsetup[table]{list=no}
										\captionsetup[code]{list=no}}
				
			%	\Figura[Posición]{Formato}{Archivo.ext}{Pie}{fig:Etiqueta}
				\newcommand{\Figura}[5][]{	
										\begin{figure}[#1]
											\centering
											\includegraphics[#2]{#3}
											\caption{#4}
											\label{#5}
										\end{figure}}
										
			%	\Figura[Posición]{Formato}{Archivo.ext}{Pie}{Pie abreviado}{fig:Etiqueta}						
				\newcommand{\FiguraPieAbreviado}[6][]{	
										\begin{figure}[#1]
											\centering
											\includegraphics[#2]{#3}
											\caption[#5]{#4}
											\label{#6}
										\end{figure}}
										
			%	\Tabla[Posición]{Formato de las columnas}{Tabla}{Pie}{tab:Etiqueta}			
				\newcommand{\Tabla}[5][]{
										\begin{table}[#1]
											\centering
											\begin{tabular}{#2}
												#3
											\end{tabular}
											\caption{#4}
											\label{#5}
										\end {table}}
										
			%	\Tabla[Posición]{Formato de las columnas}{Tabla}{Pie}{Pie abreviado}{tab:Etiqueta}			
				\newcommand{\TablaPieAbreviado}[6][]{
										\begin{table}[#1]
											\centering
											\begin{tabular}{#2}
												#3
											\end{tabular}
											\caption[#5]{#4}
											\label{#6}
										\end {table}}
				
			% \CuadroTexto{Texto}					
				\newcommand{\CuadroTexto}[1]{
										\begin{cuadrotexto}
											#1
										\end{cuadrotexto}
										\vspace{-10pt}}
										
			% \Ecuacion{Ecuación}{eq:Etiqueta}					
				\newcommand{\Ecuacion}[2]{
										\begin{equation}
											#1
											\label{#2}
										\end{equation}}
			
			% \InsertarCodigo{Lenguaje}{Ruta del archivo}				
				\newcommand{\InsertarCodigo}[2]{#1 \lstinputlisting{#2}}
				
			% \Datasheet[Parámetros]{Archivo.pdf}{Título}
				\newcommand{\Datasheet}[3][]{
										\hc{#3}
										\includepdf[#1]{#2}}
				
			% \Plano{Archivo.pdf}{Título}{pl:Etiqueta}
				\newcommand{\Plano}[3]{
										\includepdf[pages=1, offset=0 -13, addtotoc={1, chapter, 1, #2, #3}]{#1}}
																				
% ---------------------------------------------------------------------
% ----------------------------- DOCUMENTO -----------------------------
% ---------------------------------------------------------------------

\begin{document}

	\frontmatter
%		\EncabezadoCorto
%		\SuprimirIndices
	
%		\clearpage
\thispagestyle{empty}
\renewcommand{\ULthickness}{1pt}

\begin{adjustwidth}{-1.25cm}{0.12cm}	% Si se cambian los m�rgenes izquierdo y derecho del documento hay que modificar estos valores 
	\vspace*{63.5pt}
	{\fontfamily{lmss}\Huge\bfseries\selectfont{�ndice de documentos}}
	\vspace{40pt}
	
	\begin{tabbing}
		{\fontfamily{ppl}\large\bfseries\selectfont\MakeUppercase{\uline{Documento I. Memoria}}}\\
		Parte I. Memoria 			\hspace{5cm}\= p�g. \pageref*{InicioMem} \ a \ \pageref*{FinMem} 	\hspace{2.5cm}\= ?? p�ginas\\
		Parte II. Estudio econ�mico 			\> p�g. \pageref*{InicioEE} \ a \ \pageref*{FinEE}  								\> ?? p�ginas\\
		Parte III. Manual de usuario 			\> p�g. \pageref*{InicioMU} \ a \ \pageref*{FinMU}									\> ?? p�ginas\\
		Parte IV. C�digo fuente 					\> p�g. \pageref*{InicioCF} \ a \ \pageref*{FinCF} 									\> ?? p�ginas\\
		Parte V. Hojas de caracter�sticas \> p�g. \pageref*{InicioHC} \ a \ ??																\> ?? p�ginas\\\\
		
		{\fontfamily{ppl}\large\bfseries\selectfont\MakeUppercase{\uline{Documento II. Planos}}}\\
		1. Lista de planos 								\> p�g. \pageref*{InicioLP}																			\> \ \ 1 p�gina\\
		2. Planos													\> p�g. \pageref*{InicioPlanos} \ a \ ??												\> ?? p�ginas\\\\
			
		{\fontfamily{ppl}\large\bfseries\selectfont\MakeUppercase{\uline{Documento III. Pliego de condiciones}}}\\
		1. Generales y econ�micas				 	\> p�g. \pageref*{InicioCGE} \  a \ \pageref*{FinCGE}  							\> ?? p�ginas\\
		2. T�cnicas y particulares 				\> p�g. \pageref*{InicioCTP} \ a \ \pageref*{FinCTP}   							\> ?? p�ginas\\\\
			
		{\fontfamily{ppl}\large\bfseries\selectfont\MakeUppercase{\uline{Documento IV. Presupuesto}}}\\
		1. Mediciones 										\> p�g. \pageref*{InicioMed} \ a \ \pageref*{FinMed}  							\> ?? p�ginas\\
		2. Precios unitarios 							\> p�g. \pageref*{InicioPU} \ a \ \pageref*{FinPU} 									\> ?? p�ginas\\
		3. Sumas parciales 								\> p�g. \pageref*{InicioSP} \ a \ \pageref*{FinSP}  								\> ?? p�ginas\\
		4. Presupuesto general 						\> p�g. \pageref*{InicioPG} \ a \ \pageref*{FinPG}									\> ?? p�ginas\\
	\end{tabbing}
\end{adjustwidth}

\renewcommand{\ULthickness}{0.4pt}
%		\clearpage
\begin{adjustwidth}{-1.75cm}{0.62cm}	% Si se cambian los m�rgenes izquierdo y derecho del documento hay que modificar estos valores 
\thispagestyle{empty}

\vspace*{\fill}
\centering\fbox{\begin{minipage}{15cm}
\begin{center}
\vspace* {1.5cm}
Autorizada la entrega del proyecto del alumno:\\
\vspace {1cm}
\textbf{\Autor}\\

\vspace{1.5cm}
\rule{2cm}{1.2pt}
\vspace{1.8cm}

\textsc{El Director del Proyecto}\\
\vspace {1cm}
\textbf{\Director}\\
\vspace {2cm}
Fdo.: \ \ldots \ldots \ldots \ldots \ldots \ldots \ldots \ldots \hspace{1cm}
Fecha: \ \ldots \ldots \ \ / \ \ldots \ldots \ \ / \ \ldots \ldots \ldots\\

\vspace{1.5cm}
\rule{2cm}{1.2pt}
\vspace{1.8cm}
 
\textsc{V� B� del Coordinador de Proyectos}\\
\vspace {1cm}
\textbf{\Coordinador}\\
\vspace {2cm}
Fdo.: \ \ldots \ldots \ldots \ldots \ldots \ldots \ldots \ldots \hspace{1cm}
Fecha: \ \ldots \ldots \ \ / \ \ldots \ldots \ \ / \ \ldots \ldots \ldots\\
\vspace* {1.5cm}
\end{center}
\end{minipage}}
\vspace*{\fill}
\end{adjustwidth}

%		\clearpage
\begin{adjustwidth}{-2.5cm}{0cm}	% Si se cambian los m�rgenes izquierdo y derecho del documento hay que modificar estos valores 
\begin{center}
\thispagestyle{empty}

% Encabezado
	\begin{minipage}{3.2cm}
	\begin{flushleft}
	\includegraphics{Logo.pdf}
	\end{flushleft}
	\end{minipage}
	\ \ 
	\hfill\begin{minipage}{12.85cm}
	\begin{center}
	\fontsize{20}{20}\selectfont \MakeUppercase{Universidad Pontificia Comillas}\\
	\vspace{0.3cm}
	\fontsize{14}{14}\selectfont\textbf{\MakeUppercase{Escuela T�cnica Superior de Ingenier�a (ICAI)}}\\
	\vspace{0.1cm}
	\normalsize\textbf{\MakeUppercase{\Titulacion}}\\
	\end{center}
	\end{minipage}

\vspace {7.92cm}

% T�tulo
	\Large\MakeUppercase{Proyecto Fin de Carrera}\\
	\vspace{0.8cm}
	\LARGE\textbf{\MakeUppercase{\Titulo}}\\
	\end{center}

\vspace{7.92cm}

% Autor
	\normalsize
	\begin{flushright}
	\textbf{\MakeUppercase{Autor:} \Autor}\\
	\vspace{0.2cm}
	\textbf{\MakeUppercase{Director:} \Director}\\
	\vspace{0.4cm}
	\textbf{\MakeUppercase{Madrid}, \Fecha}
	\end{flushright}
	
\end{adjustwidth}
		\input{abstract-spanish.tex}
		\input{abstract-english.tex}
%		\clearpage
\begin{adjustwidth}{-1.75cm}{0.75cm}

\thispagestyle{empty}

\vspace* {5cm}
\begin{flushright}
\textit{Aqu� se puede poner\\ una dedicatoria si se desea.}\\
\vspace {10cm}
\textit{Si se quiere poner una cita,\\ este es el espacio reservado para ello.}\\
\textsc{Autor de la cita}\\
\vspace {1cm}
\textit{Este espacio se puede usar para\\ escribir una segunda cita.}\\
\textsc{Autor de la cita}
\end{flushright}
\end{adjustwidth}
%		\selectlanguage{spanish}
\clearpage
\pdfbookmark[0]{Agradecimientos}{Agradecimientos}
\chapter*{Agradecimientos}
\leftside{Agradecimientos}


	
	\mainmatter
%		\EncabezadoLargo
%		\ActualizarIndices
		
	% Documento I. Memoria
%		\doc{Memoria}
%			\IndiceDocumento{Índice}{IndiceMemoria}
% 			\IndiceFiguras
%				\IndiceTablas
%				\IndiceCodigo
				\input{acronyms.tex}
				\clearpage
\pdfbookmark[0]{Símbolos}{Simbolos}
\chapter*{Símbolos}

\noindent
\begin{xtabular}{@{}p{1.25cm}@{}l}
\shrinkheight{-155pt}
$\alpha$				& Primera letra del alfabeto griego \\
$\beta$					& Segunda letra del alfabeto griego \\
\end{xtabular}
				
%				\part{Memoria}{}{InicioMem}
					\chapter {Primeros pasos}
%\lettrine{E}{ste} documento es una plantilla de \LaTeX \ para escribir el Proyecto Fin de Carrera. Trae creados por defecto la mayoría de los apartados que suelen ser habituales en un documento de este tipo --si bien es posible que muchos usuarios no los necesiten todos-- e incorpora unas macros para facilitar su uso. Dichas macros están explicadas en más detalle en el capítulo siguiente. Es importante destacar, sin embargo, que en ningún caso pretende ser un manual de \LaTeX, por lo que se precisan unos conocimientos mínimos para poder usarla correctamente.

\section{Software necesario}
En primer lugar, es requisito imprescindible disponer de una distribución de \LaTeX, un compilador y un visor de documentos PDF, que variarán en función del sistema operativo (Windows, MacOS o Linux). Dado que la mayoría de los usuarios trabajarán bajo Windows, nos centraremos en este sistema. No obstante, es sencillo encontrar en Internet las distribuciones y compiladores apropiados para MacOS y Linux.

En cuanto a la \textbf{distribución} se recomienda instalar \href{http://miktex.org/2.8/setup}{MiKTeX}. Si se dispone de Windows 7, la versión recomendada es la 2.8. Sin embargo, a veces presenta problemas en sistemas operativos anteriores como Vista o XP, por lo que en esos casos lo más seguro es emplear MiKTeX 2.7. Existen multitud de \textbf{compiladores} para Windows, algunos son \textit{software} libre y otros propietario. Esta plantilla se ha escrito con \href{http://www.texniccenter.org}{TeXnicCenter}, un compilador gratuito bastante fácil de manejar. Finalmente, en lo que se refiere al \textbf{visor de PDF}, hay muchos donde elegir (Adobe Reader, SumatraPDF...).

Una vez instalado todo esto, se puede abrir la plantilla a través del archivo con extensión \texttt{\small .tcp}, que es un proyecto de TeXnicCenter. Si ha instalado otro compilador, tenga en cuenta que el documento principal es \texttt{\small main.tex}.

\begin{comment}

\section{Familiarizándose con \LaTeX \ y la plantilla}
Se sugiere, antes de empezar a escribir, leer la plantilla en código \LaTeX, además de en PDF, para conocer los comandos disponibles y porque durante la redacción de la misma se han empleado los de uso más frecuente. Preste especial atención a los comentarios. Si desea profundizar en las funcionalidades de algún paquete concreto puede consultarlo fácilmente en Internet. Para los usuarios noveles también se recomienda leer, o al menos hojear, \href{http://www.ctan.org/tex-archive/info/lshort/english/lshort.pdf}{\textit{The Not So Short Introduction to \LaTeX \ 2$\varepsilon$}}.

Mientras seguía esta recomendación habrá observado que existe un apartado en el documento \texttt{\small main.tex} que se denomina \textbf{Parámetros}. No olvide cambiar los valores para que se adapten a sus necesidades. \LaTeX \ actualizará automáticamente todos los campos que los contengan la próxima vez que compile.

\section{Observaciones importantes}
Como ya se habrá dado cuenta, la plantilla automatiza gran parte de las tareas. Sin embargo, el usuario debe cuidar algunos aspectos por su cuenta.
\begin{itemize}
	\item En el \textbf{Índice de documentos}, que está en el archivo \texttt{\small Contraportada.tex} hay que sustituir las ?? por los números de página correspondientes cuando haya terminado de escribir el Proyecto. Es posible que necesite modificar algunas de las entradas si no cuenta con todas las divisiones.
	\item La \textbf{Hoja de autorización} también puede requerir modificaciones si tiene más de un director.
	\item En la \textbf{Portada} puede ser necesario ajustar el espacio vertical si el título no ocupa dos líneas.
	\item Si modifica los \textbf{márgenes} actuales, deberá ajustar en consecuencia los parámetros del comando {\small\verb|\adjustwidth|} en \texttt{\small Contraportada.tex}, \texttt{\small Autorización.tex}, y \texttt{\small Portada.tex}.
	\item En \texttt{\small Dedicatoria.tex} puede querer modificar el formato de la dedicatoria y las citas.
	\item No se olvide de modificar las listas de \textbf{Acrónimos} y \textbf{Símbolos}.
	\item Cuando compile, hágalo tantas veces como sea necesario hasta que el número de \textsl{Warnings} sea constante. Eso significa que todas las referencias cruzadas están actualizadas.
	\item Habrá observado que los niveles de los marcadores PDF no son del todo correctos. Cuando tenga la versión definitiva, si tiene instalado Adobe Acrobat Professional puede cambiarlo fácilmente arrastrándolos hasta el lugar deseado. En caso contrario, deberá editar el archivo \texttt{\small ProyectoFinCarrera.out}.\\
	Las modificaciones consisten en cambiar el nivel de los \textbf{Documentos} de \texttt{\small[0]} a \texttt{\small[-1]} y escribir, en el último campo de las demás referencias de nivel 0, el primer campo entre \{\} del documento inmediatamente superior, para indicar la subordinación. Para mantener los cambios hay que evitar que se actualicen los marcadores añadiendo la siguiente línea al final del archivo:
\begin{quote}
\scriptsize\begin {verbatim}
\let\WriteBookmarks\relax
\end {verbatim}
\normalsize
\end{quote}
Por ejemplo, habría que reemplazar:
\begin{quote}
\scriptsize\begin {verbatim}
\BOOKMARK [0][-]{section*.4}{Documento I. Memoria}{}
\BOOKMARK [0][-]{IndiceMemoria.0}{\315ndice}{}
\end {verbatim}
\normalsize
\end{quote}
por esto otro:
\begin{quote}
\scriptsize\begin {verbatim}
\BOOKMARK [-1][-]{section*.4}{Documento I. Memoria}{}
\BOOKMARK [0][-]{IndiceMemoria.0}{\315ndice}{section*.4}
\end {verbatim}
\normalsize
\end{quote}
\end{itemize}

\end{comment}
					\chapter{Macros}
%\lettrine{E}{ste} capítulo pretende explicar brevemente las macros que se han definido para hacer más intuitivo el uso de la plantilla, con el fin de evitar, en la medida de lo posible, usar código muy complejo que, si bien es necesario, no aporta nada relevante a un usuario convencional.

\begin{comment}

\section {Figuras}
Probablemente lo más importante en un documento de las características de un Proyecto Fin de Carrera son las figuras, dado que permiten presentar mucha información de forma clara y esquemática. Para insertar una imagen se debe emplear la siguiente macro:
\begin{quote}
\scriptsize\begin {verbatim}
\Figura[Posicion]{Formato}{Archivo.ext}{Pie}{fig:Etiqueta}

\Figura[H]{scale = 0.45, angle = 0}{LaTeXLogo.png}
          {Ejemplo de como insertar una figura}{fig:EjemploFigura}
\end {verbatim}
\normalsize
\end{quote}

\Figura[H]{scale = 0.45, angle = 0}{LaTeXLogo.pdf}{Ejemplo de cómo insertar una figura}{fig:EjemploFigura}

Los campos significan lo siguiente:
\begin{itemize}
	\item \textbf{Posición:} Indica dónde se quiere colocar la imagen. Este argumento es opcional y puede omitirse si no se escriben los corchetes ([]). En ese caso, \LaTeX \ colocará la imagen donde estime más conveniente. Si se quiere controlar explícitamente el posicionamiento, las posibilidades son:
		\begin{itemize}
		\item \textbf{H:} \LaTeX \ hará todo lo posible para que la imagen esté exactamente en esta posición. Téngase en cuenta que en algunos casos puede ocasionar algún problema de formato.
		\item \textbf{h:} Especifica que la imagen debe estar colocada en esta posición, pero si \LaTeX \ encuentra dificultades puede desplazarla.
		\item \textbf{t:} La imagen debe ir en la parte superior de la página.
		\item \textbf{b:} La imagen debe ir en la parte inferior de la página.
	\end{itemize}
	\item \textbf{Formato:} Permite configurar el formato de las imágenes, indicando la escala respecto al tamaño original, la inclinación...
	\item \textbf{Nombre del archivo:} Debe incluir la \underline{extensión} y preferiblemente su nombre no debe contener espacios ni tildes. El espacio de caracteres recomendado es UNIX. La imagen debe estar ubicada en la carpeta \textsl{Imágenes} incluida junto con los archivos del proyecto. En principio admite cualquier formato típico (.jpg, .png, y .gif entre otros), aunque se recomienda incluir las imágenes en formato .pdf\footnote{Para guardar una imagen como .pdf debe usar alguna de la múltiples impresoras de PDF que existen, bien la propietaria de Adobe o alguna versión gratuita.} o .eps\footnote{Puede convertir imágenes a .eps con programas como Gimp o Adobe Photoshop.} porque el visor de documentos Adobe Reader (con otros como el SumatraPDF no sucede) muestra las letras de las páginas donde se incluyen imágenes en otro formato en negrita, si bien a la hora de imprimir lo hace correctamente.
	\item \textbf{Pie de figura:} Texto que ha de aparecer al pie de la figura. La etiqueta Figura X. la introduce la macro automáticamente.
	\item \textbf{Etiqueta:} Permite referenciar la imagen posteriormente dentro del documento. Se recomienda empezarlas con el prefijo \texttt{\small fig:} para distinguir las etiquetas de otros entornos. No pueden contener espacios. Para referenciar un objeto se pueden usar los comandos {\small\verb|\ref{fig:EjemploFigura}|} o {\small\verb|\autoref{fig:EjemploFigura}|}, que insertan bien el número del objeto (\ref{fig:EjemploFigura}), bien el nombre y el número (\autoref{fig:EjemploFigura}) respectivamente.
\end{itemize}

\section {Tablas}
Para recoger y comparar datos, las tablas suelen ser muy útiles. La macro para crearlas es:
\begin{quote}
\scriptsize\begin {verbatim}
\Tabla[Posición]{Formato de las columnas}{Tabla}{Pie}{tab:Etiqueta}

\Tabla[H]{|l|c|r|}{
\hline
\multicolumn{3}{|c|}{Encabezado}\\ \hline
Columna 1 & Columna 2 & Columna 3\\ \hline
Fila 1		& 1.258 		& 1.025.258\\
Fila 2 		& 15.678 		& 159\\ \hline
}{Ejemplo de cómo insertar una tabla}{tab:EjemploTabla}
\end {verbatim}
\normalsize
\end{quote}

\Tabla[H]{|l|c|r|}{
\hline
\multicolumn{3}{|c|}{Encabezado}\\ \hline
Columna 1 & Columna 2 & Columna 3\\ \hline
Fila 1		& 1.258 		& 1.025.258\\
Fila 2 		& 15.678 		& 159\\ \hline
}{Ejemplo de cómo insertar una tabla}{tab:EjemploTabla}

Los campos que son diferentes a los explicados en el apartado anterior son:
\begin{itemize}
	\item \textbf{Formato de las columnas:} Determina cuantas columnas tendrá la tabla y si el texto irá alineado a la izquierda (\texttt{l}), a la derecha (\texttt{r}) o centrado (\texttt{c}) en la misma. El símbolo | entre dos columnas indica una línea vertical separadora. Si se colocan dos, entonces se mostrará una doble línea vertical.
	\item \textbf{Tabla:} Define el formato y el contenido de la tabla. Las columnas se separan por \texttt{\&} y al final de una fila se debe insertar una doble barra para indicar el salto de línea, como se puede observar en el ejemplo. Para insertar una línea horizontal se emplea el comando {\small\verb|\hline|}. Si desea obtener más información acerca de cómo personalizar las tablas se recomienda consultar el \href{http://en.wikibooks.org/wiki/LaTeX/}{\textsl{wikilibro} de \LaTeX}.
\end{itemize}

\section{Ecuaciones}
Una de las grandes ventajas de \LaTeX \ es su facilidad para escribir ecuaciones. La macro en este caso es:
\begin{quote}
\scriptsize\begin {verbatim}
\Ecuacion{Ecuación}{eq:Etiqueta}

\Ecuacion{\mu^i_{K_t} = \sum_{j=1}^{K_t}\frac{x^i_j}{K_i}}{eq:EjemploEcuacion}
\end {verbatim}
\normalsize
\end{quote}

\Ecuacion{\mu^i_{K_t} = \sum_{j=1}^{K_t}\frac{x^i_j}{K_i}}{eq:EjemploEcuacion}

Para aquellos usuarios no familiarizados con el formato de introducción de ecuaciones, existen infinidad de editores \textit{online} que generan código \LaTeX \ a partir de una interfaz similar al editor de ecuaciones de Windows. Uno bastante intuitivo es el siguiente: \href{http://www.codecogs.com/latex/eqneditor.php}{http://www.codecogs.com/latex/eqneditor.php}.

Si se quiere insertar una ecuación entre el texto se ha de escribir entre símbolos de dólar (\texttt{\$}). Una ecuación entre el texto queda así: $\sen\alpha = \cos\varphi$.

\section{Código}
Es frecuente que en los proyectos sea necesario mostrar durante el desarrollo de la memoria extractos de código para ilustrar el hilo de la exposición o para explicar su funcionamiento. Desafortunadamente, no resulta sencillo hacer una macro que simplifique la inclusión de código, de modo que cuando se requiera se deberá copiar y editar adecuadamente el siguiente ejemplo:
\begin{quote}
\scriptsize\begin{verbatim}
\begin{code}[Posición]
\capstart
\begin{cuadrotexto}
\VisualCpp	% Lenguaje. Opciones disponibles: \VisualCpp y \Matlab
\begin{lstlisting}
Código
\end{lstlisting}
\end{cuadrotexto}
\vspace*{-10pt}
\caption{Pie}
\label{cod:Etiqueta}
\end{code}

\begin{code}[H]
\capstart
\begin{cuadrotexto}
\VisualCpp	% Lenguaje. Opciones disponibles: \VisualCpp y \Matlab
\begin{lstlisting}
int main(void)
{
	while (1);
}
\end{lstlisting}
\end{cuadrotexto}
\vspace*{-10pt}
\caption{Ejemplo de cómo insertar un extracto de código}
\label{cod:EjemploCodigo}
\end{code}
\end {verbatim}
\normalsize
\end{quote}

\begin{code}[H]
\capstart
\begin{cuadrotexto}
\VisualCpp	% Lenguaje. Opciones disponibles: \VisualCpp y \Matlab (Se pueden programar otras si es necesario)
\begin{lstlisting}
int main(void)
{
	while (1);
}
\end{lstlisting}
\end{cuadrotexto}
\vspace*{-10pt}
\caption{Ejemplo de cómo insertar un extracto de código}
\label{cod:EjemploCodigo}
\end{code}

Para insertar archivos de código completos, en la Parte de Código fuente, por ejemplo, se puede usar la macro:
\begin{quote}
\scriptsize\begin{verbatim}
\InsertarCodigo{Lenguaje}{Ruta del archivo}

\InsertarCodigo{\VisualCpp}{Codigo/main.c}
\end {verbatim}
\normalsize
\end{quote}

\section{Cuadros de texto}
Si se quiere insertar un cuadro de texto para destacar alguna información, el código es el siguiente:
\begin{quote}
\scriptsize\begin{verbatim}
\CuadroTexto{Texto}
\end{verbatim}
\normalsize
\end{quote}

\CuadroTexto{Texto}

\section{Bibliografía}
Evidentemente en un Proyecto Fin de Carrera es imprescindible referenciar la información. Para introducir una cita bibliográfica, en primer lugar hay que crear la entrada en la Bibliografía. Un posible formato para artículos, libros y proyectos sería:

\begin{quote}
\scriptsize\begin {verbatim}
\bibitem{bib:Etiqueta} 
\textbf{Autor}, 
\textit{Título},
Revista, editorial, fecha...

\bibitem{bib:LaTeX} 
\textbf{Oetiker, T., Partl, H., Hyna, I., Schlegl, E.}, 
\textit{The Not So Short Introduction to \LaTeX \ 2$\varepsilon$}.
Diciembre 2009.
\end {verbatim}
\normalsize
\end{quote}

Mientras que para páginas web se podría incluir de la siguiente manera:
\begin{quote}
\scriptsize\begin {verbatim}
\bibitem{www:Etiqueta} 
\textbf{Título},
\textit{Descripción}.
Última consulta: Fecha\\
\href{http://www.iit.upcomillas.es/pfc}{http://www.iit.upcomillas.es/pfc}

\bibitem{www:PFC} 
\textbf{Universidad Pontificia Comillas},
\textit{Página web de Proyectos Fin de Carrera}.
Última consulta: 29/04/2010\\
\href{http://www.iit.upcomillas.es/pfc}{http://www.iit.upcomillas.es/pfc}
\end {verbatim}
\normalsize
\end{quote}

Finalmente, para referenciar un documento en el texto se usaría:
\begin{quote}
\scriptsize\begin {verbatim}
\cite{Etiqueta}

\cite{bib:LaTeX}

\cite{Etiqueta,Etiqueta}

\cite{bib:LaTeX,www:PFC}
\end {verbatim}
\normalsize
\end{quote}

El resultado para una única cita sería \cite{bib:LaTeX} y para citas múltiples \cite{bib:LaTeX,www:PFC}.

\section{Notas al pie}
En algún momento puede ser útil insertar una nota al pie. Para ello el código es:
\begin{quote}
\scriptsize\begin {verbatim}
\footnote{Nota al pie}
\end {verbatim}
\normalsize
\end{quote}

El resultado sería el siguiente\footnote{Nota al pie}.

\section{Otros comandos}
Si ha leído la plantilla en código \LaTeX, como se recomendó previamente, habrá encontrado ejemplos de uso de otros comandos que pueden resultar útiles.

\end{comment}

					\input{chapter03.tex}
%					\clearpage
\phantomsection
\addcontentsline{toc}{chapter}{Bibliograf�a}
\rightside{Bibliograf�a}

\begin{thebibliography}{99}

		\bibitem{bib:LaTeX} 
		\textbf{Oetiker, T., Partl, H., Hyna, I., Schlegl, E.}, 
		\textit{The Not So Short Introduction to \LaTeX \ 2$\varepsilon$},
		Diciembre 2009.
		
		\bibitem{www:PFC} 
		\textbf{Universidad Pontificia Comillas},
		\textit{P�gina web de Proyectos Fin de Carrera}.
		�ltima consulta: 29/04/2010\\
		\href{http://www.iit.upcomillas.es/pfc}{http://www.iit.upcomillas.es/pfc}
				
\end{thebibliography}
%				\part{Estudio econ�mico}{FinMem}{InicioEE}
\EncabezadoCorto

%				\part{Manual de usuario}{FinEE}{InicioMU}
\EncabezadoLargo

%				\label{FinHC}
%			\FinIndiceDocumento
		
%			\SuprimirIndices
			
	% Documento II. Planos	
%		\doc{Planos}
%			\ListaPlanos{Lista de planos}{InicioLP}
%			\label{InicioPlanos}
%				\Plano{Planos/Plano.pdf}{Ejemplo de c�mo insertar un plano}{pl:EjemploPlano}
%			\label{FinPlanos}
%			\FinIndiceDocumento
			
	% Documento III. Pliego de condiciones			
%		\doc{Pliego de condiciones}
%			\IndiceDocumento{Índice}{IndicePliego}
%				\chapter{Condiciones generales y econ�micas}
\label{InicioCGE}
% Aqu� va el texto
\label{FinCGE}

\chapter{Condiciones t�cnicas y particulares}
\label{InicioCTP}
% Aqu� va el texto
\label{FinCTP}
%			\FinIndiceDocumento
			
	% Documento IV. Presupuesto	
%		\doc{Presupuesto}
%			\IndiceDocumento{Índice}{IndicePresupuesto}
%				\chapter{Mediciones}
\label{InicioMed}
% Aqu� va el texto
\label{FinMed}

\chapter{Precios unitarios}
\label{InicioPU}
% Aqu� va el texto
\label{FinPU}

\chapter{Sumas parciales}
\label{InicioSP}
% Aqu� va el texto
\label{FinSP}

\chapter{Presupuesto general}
\label{InicioPG}
% Aqu� va el texto
\label{FinPG}
%			\FinIndiceDocumento	

\end{document}
